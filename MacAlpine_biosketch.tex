%!TEX TS-program = xelatex
\documentclass{nihbiosketch}

%\usepackage{draftwatermark}  % delete this in your document!
%\SetWatermarkText{Sample}    % delete this in your document!
%\SetWatermarkLightness{0.9}  % delete this in your document!

%------------------------------------------------------------------------------

% Biological Abbreviations
\newcommand\dros{{\itshape Drosophila}\xspace}
\newcommand\dmel{{\itshape D.~melanogaster}\xspace}
\newcommand\scer{{\itshape S.~cerevisiae}\xspace}
\newcommand\saccer{{\itshape Saccharomyces cerevisiae}\xspace}
\newcommand\xenopus{{\itshape Xenopus}\xspace}
\newcommand\invitro{{\itshape in~vitro}\xspace}
\newcommand\invivo{{\itshape in~vivo}\xspace}
\newcommand\ten[1]{$\times$10$^{#1}$}
\newcommand\panc{Panc\dash1\xspace}
\newcommand\orcmt{\emph{orc1\dash 161}\xspace}

% Other Abbreviations
\newcommand\eg{\emph{e.g.}\xspace}
\newcommand\dash{\nobreakdash-\hspace{0pt}}

\name{MacAlpine, David M.}
\eracommons{MACALP001}
\position{Associate Professor of Pharmacology and Cancer Biology}

\begin{document}
%------------------------------------------------------------------------------

\begin{education}
University of Texas A\&M  & BS           & 1993  & Biochemistry \\
University of Texas A\&M  & MS           & 1995  & Genetics \\
University of Texas Southwestern Med. Ctr.  & PhD & 2001 & Genetics\\
Massachusetts Institute of Technology  & Postdoctoral  & 2006 & Biochemistry \\
\end{education}


\section{Personal Statement}
\begin{statement}
\addtolength{\parskip}{1.5mm}
\noindent At each stage of my career development, I have been fascinated by how two fundamental processes -- DNA replication and transcription -- are coordinated to maintain the expression and inheritance of genetic information.  As a research assistant at Texas A\&M University with Geoffery Kapler, I identified and mapped developmentally regulated replication pause sites in the rDNA of the ciliate, \textit{Tetrahymena thermophila}.  As a graduate student with Ron Butow and Phil Perlman at the University of Texas Southwestern Medical Center, I deciphered the molecular mechanisms by which yeast hypersuppressive petite mitochondrial genomes are preferentially inherited over wild type mitochondrial genomes.  As a Damon Runyon fellow in Stephen Bell's laboratory at the Massachusetts Institute of Technology, I developed a systematic approach using genomic tiling arrays to characterize the replication and transcription programs of a \dros\ chromosome. Finally, as an independent investigator at Duke University, my research program has been focused on understanding how the start sites of DNA replication are selected and regulated, in the context of the local chromatin environment, to maintain genomic stability and to ensure the accurate inheritance of genetic and epigenetic information. We utilize a multidisciplinary approach combining genetics, biochemistry, and cell biology with genomics and computational biology to systematically dissect the regulatory elements and mechanisms that direct the DNA replication program in two eukaryotic model systems \dmel \textsuperscript{a-c} and \scer \textsuperscript{d}.

\noindent My research group consists of both experimental and computer scientists and I have trained and supervised dissertation students from a number of graduate programs at Duke, including Pharmacology (1), Computational Biology and Bioinformatics (2), Molecular Cancer Biology (2), and Genetics and Genomics (2).   I have a long standing interest in mentoring students from diverse backgrounds.  To this end, I am the Director of Graduate studies in Pharmacology, an undergraduate advisor for Duke freshmen and sophomores, a faculty mentor for Duke's IMSD sponsored BioCORE program, and I am currently mentoring an F31 diversity award recipient, Ms. Monica Gutierrez, in my laboratory.

\noindent In addition to my group's research activity, I am actively involved in service to the research community.  I frequently participate in peer review, and I serve on the editorial boards of \textit{Genome Research} and \textit{BMC Genomics}.  I served as permanent reviewer for the American Cancer Society (DNA Mechanisms of Cancer) and I am currently a standing member of the NIH Molecular Genetics A (MGA) study section.  I have also been an \emph{ad hoc} grant reviewer for various special emphasis review panels including multiple phases of NHGRI's ENCODE and the 4D nucleome project. I have also participated in an intramural site visit review for the Laboratory of Molecular Pharmacology (NCI).

 




%Use the Personal Statement to address MIRA specific elements of the Investigator Review Criterion.  Include descriptions of significant service and contributions to the research community. Under the section on research contributions, emphasize especially contributions over the past five years.  Be sure to include a link to your My Bibliography complete list of publications and be sure the data are complete and up to date.

%Are the PD(s)/PI(s), collaborators, and other researchers well suited to the project? Has the PI demonstrated an ongoing record of accomplishments that have advanced their field(s)? If the research is collaborative, do the investigators have complementary and integrated expertise?

%For this FOA:  Does the PD/PI have a consistent and current record of productivity and scientific impact commensurate with the applicant's current career stage? Is he/she clearly the intellectual driver of the proposed research program? Has the PD/PI shown evidence of being creative and adaptable, able to recognize new opportunities and explore new areas of scientific inquiry, and open to the use of new systems and strategies, as appropriate for the proposed research program? Does the PD/PI have a record of conducting and reporting rigorous, reproducible, transparent, and cost-effective research? Does the PD/PI have a record of significant service and contributions to the research community?





%My research program exploits multiple eukaryotic model systems and a multidisciplinary approach combining experimental and computational biology to understand how the local chromatin environment establishes and regulates the DNA replication program.


%My group,  together with our collaborators,  have made fundamental contributions to our understanding of the DNA replication program in bacteria, yeast, fruit flies, and mammalian systems.  We utilize an interdisciplinary approach combining genetics, biochemistry, and cell biology with genomics and computational biology to systematically dissect the regulatory elements and mechanisms that direct the DNA replication program.  We have found that the local chromatin architecture is a primary determinant of eukaryotic origin selection and activation. 





%Based on these technologies, I (along with my collaborators) have helped to pioneer the application of genomic approaches to study DNA replication in bacteria, S. cerevisiae, Drosophila and human cell lines.  My independent research at Duke University has exploited multiple eukaryotic model systems and a multidisciplinary approach combining experimental and computational biology to understand how the local chromatin environment establishes and regulates the DNA replication program.



% the more than six decades since the discovery of the double helix, there has been tremendous progress in understanding the biochemical mechanisms responsible for the precise and accurate duplication of the genome. However, despite the remarkable conservation of proteins and protein functions required for DNA replication across both prokaryotic and eukaryotic systems, we know comparatively little about the functional elements that direct DNA replication in higher eukaryotes.  My research program is focused on understanding how the start sites of DNA replication are selected and regulated in the context of the local chromatin environment to maintain genomic stability and to ensure the accurate inheritance of genetic and epigenetic information.  Failure to accurately regulate DNA replication may result in under or over replication of the genome and lead to genomic instability -- a hallmark of tumorigenesis.   In addition, mutations in multiple replication initiation factors have recently been linked to Meier-Gorlin syndrome, a primordial form of dwarfism, indicating a critical role for the DNA replication program in development.  As a postdoctoral fellow, I helped pioneer the use of genomic approaches to study the eukaryotic DNA replication program of the model organism Drosophila melanogaster.  My group,  together with our collaborators,  have made fundamental contributions to our understanding of the DNA replication program in bacteria, yeast, fruit flies, and mammalian systems.  We utilize an interdisciplinary approach combining genetics, biochemistry, and cell biology with genomics and computational biology to systematically dissect the regulatory elements and mechanisms that direct the DNA replication program.  We have found that the local chromatin architecture is a primary determinant of eukaryotic origin selection and activation.  A mechanistic understanding of how replication origins are established and regulated in proliferative cells will have far-ranging implications for developmental, stem cell, and cancer biology and may eventually provide the critical foundations for the development of future DNA based gene therapies.  


%My research group is focused on identifying and understanding the chromosomal mechniasmeby 



%the functional DNA elements that direct and regulate the DNA replication and transcription programs. My group has pioneered the use of genomic approaches to characterize the DNA replication program in eukaryotic organisms.  I was a principal investigator of a NHGRI modENCODE data production project to identify all functional DNA elements in the \textit{Drosophila} genome that direct the DNA replication program.  I was also a co-PI of the modENCODE data analysis center (DAC) and involved in the systematic and comprehensive analysis of 1000s of datasets generated from both the modENCODE and ENCODE consortia.  As a modENCODE data production laboratory we have developed robust protocols and pipelines for the generation and analysis of high-throughput genomic data including ChIP-seq, ChIP-exo, RNA-seq,NET-seq, MNase-seq and DNase-seq.  We have recently developed experimental and computational protocols to characterize protein-DNA occupancy (nucleosomes, transcription factors, chromatin remodelers) at nucleotide resolution across an entire eukaryotic genome.   My research group provides a unique training environment that utilizes model organisms, genomics, and computational biology to address fundamental questions in chromatin biology.   My group consists of both experimental and computer scientists and I have trained and supervised dissertation students from a number of graduate programs at Duke, including Pharmacology (1), Computational Biology and Bioinformatics (2), Molecular Cancer Biology (2), and Genetics and Genomics (2).   I have a long standing interest in mentoring students from diverse backgrounds.  To this end, I am the Director of Graduate studies in Pharmacology, an undergraduate advisor for Duke freshmen and sophomores, a faculty mentor for Duke's IMSD sponsored BioCORE program, and I am currently mentoring an F31 diversity award recipient, Ms. Monica Gutierrez, in my laboratory. 


\begin{enumerate}



\item Lubelsky Y, Prinz JA, DeNapoli L, Li Y, Belsky JA, \textbf{MacAlpine DM}. DNA
replication and transcription programs respond to the same chromatin cues. Genome Res. 2014 Jul;24(7):1102-14. doi: 10.1101/gr.160010.113. 

\item Ho JW, Jung YL, Liu T, Alver BH, Lee S, Ikegami K, Sohn KA, Minoda A, Tolstorukov MY, Appert A, Parker SC, Gu T, Kundaje A, Riddle NC, Bishop E, Egelhofer TA, Hu SS, Alekseyenko AA, Rechtsteiner A, Asker D, Belsky JA, Bowman SK, Chen QB, Chen RA, Day DS, Dong Y, Dose AC, Duan X, Epstein CB, Ercan S, Feingold EA, Ferrari F, Garrigues JM, Gehlenborg N, Good PJ, Haseley P, He D, Herrmann M, Hoffman MM, Jeffers TE, Kharchenko PV, Kolasinska-Zwierz P, Kotwaliwale CV, Kumar N, Langley SA, Larschan EN, Latorre I, Libbrecht MW, Lin X, Park R, Pazin MJ, Pham HN, Plachetka A, Qin B, Schwartz YB, Shoresh N, Stempor P, Vielle A, Wang C, Whittle CM, Xue H, Kingston RE, Kim JH, Bernstein BE, Dernburg AF, Pirrotta V, Kuroda MI, Noble WS, Tullius TD, Kellis M, \textbf{MacAlpine DM}*, Strome S, Elgin SC, Liu XS, Lieb JD, Ahringer J, Karpen GH, Park PJ. Comparative analysis of metazoan chromatin organization. Nature. 2014 Aug 28;512(7515):449-52. * Co-corresponding author
 

\item Powell SK, MacAlpine HK, Prinz JA, Li Y, Belsky JA, \textbf{MacAlpine DM}. Dynamic
loading and redistribution of the Mcm2-7 helicase complex through the cell cycle. EMBO J. 2015 Feb 12;34(4):531-43. doi: 10.15252/embj.201488307. Epub 2015 Jan 2. 

\item Belsky JA, MacAlpine HK, Lubelsky Y, Hartemink AJ, \textbf{MacAlpine DM}. Genome-wide chromatin footprinting reveals changes in replication origin architecture induced by pre-RC assembly. Genes Dev. 2015 Jan 15;29(2):212-24.

\end{enumerate}

\end{statement}

%------------------------------------------------------------------------------
\section{Positions and Honors}

\subsection*{Positions and Employment}
\begin{datetbl}
2006--2013  & Assistant Professor of Pharmacology and Cancer Biology, Duke University Medical Center\\
2013-- & Associate Professor of Pharmacology and Cancer Biology, Duke University Medical Center \\
2015--      & Director of Graduate Studies in Pharmacology, Duke University Medical Center \\
\end{datetbl}

%2001-2004 	Damon Runyon Cancer Research Foundation Fellowship
%2006		Assistant Professor of Pharmacology and Cancer Biology, Duke Medical %Ctr, Durham, NC
%2006-2011	Whitehead Scholar Award
%2006-2013	Member, Duke Institute for Genome Sciences & Policy
%2007-		Member, Duke Comprehensive Cancer Center
%2009-		Associate Editor, BMC Genomics
%2013-		Executive Committee of Graduate Faculty, Duke University
%2013-		Associate Professor of Pharmacology and Cancer Biology, Duke Medical %Ctr, Durham, NC
%2013		NIH, MGA Ad hoc reviewer
%2014-		Editorial Board, Genome Research
%2014-		American Cancer Society, DMC reviewer
%2015		NIH,  MGA Ad hoc reviewer
%2015-		Director of Graduate Studies in Pharmacology, Duke University
%2016-		NIH, MGA Member

\subsection*{Other Experience and Professional Memberships}
\begin{datetbl}
2007--           & Member, Duke Comprehensive Cancer Center \\
2009--           & Associate Editor, BMC Genomics\\
2006--2013     & Member, Duke Institute for Genome Sciences and Policy\\
2013--    & Executive Committee of Graduate Faculty, Duke University \\
2013    & NIH, MGA Ad hoc Reviewer \\
2014-- & Editorial Board, Genome Research \\
2014--2016 & Amercian Cancer Society, DMC Reviewer \\
2015    & NIH, MGA Ad hoc Reviewer \\
2016--  & NIH, MGA Member \\
\end{datetbl}

\subsection*{Honors}
\begin{datetbl}
2001--2004           & Damon Runyon Cancer Research Foundation Fellowship \\
2006--2011           & Whitehead Scholar Award \\
\end{datetbl}

%------------------------------------------------------------------------------

\section{Contribution to Science}

\begin{enumerate}


\item Model Organism Encyclopedia of DNA Elements (modENCODE)

I was awarded a grant from the National Human Genome Research Institute for the ‘Systematic Identification and Analysis of Replication Origins in Drosophila’ as part of the model organism ENCODE (modENCODE) initiative. The ENCODE (Encyclopedia of DNA elements) project was a consortium assembled to identify and catalog all functional DNA elements in the human, worm, and fly genomes. I was also a co-PI of the modENCODE data analysis center (DAC) and involved in the systematic and comprehensive analysis of thousands of datasets generated from both the modENCODE and ENCODE consortia.  As a modENCODE data production laboratory we developed robust protocols and pipelines for the generation and analysis of high-throughput genomic data including ChIP-seq, ChIP-exo, RNA-seq, MNase-seq and DNase-seq.  Together, the consortium generated almost 3,000 publicly available genomic data sets, consisting of array and sequencing based experiments describing the transcription program, mapping the chromatin landscape, identifying transcription factor and insulator binding sites, and characterizing the DNA replication program across multiple cell lines and developmental stages.  The collaborative nature of the consortium environment led to a number of high impact discoveries that span multiple fields and disciplines\textsuperscript{a-d}.

  
\begin{enumerate}
\setlength\itemsep{0.35em}

\item Ho JW, Jung YL, Liu T, Alver BH, Lee S, Ikegami K, Sohn KA, Minoda A, Tolstorukov MY, Appert A, Parker SC, Gu T, Kundaje A, Riddle NC, Bishop E, Egelhofer TA, Hu SS, Alekseyenko AA, Rechtsteiner A, Asker D, Belsky JA, Bowman SK, Chen QB, Chen RA, Day DS, Dong Y, Dose AC, Duan X, Epstein CB, Ercan S, Feingold EA, Ferrari F, Garrigues JM, Gehlenborg N, Good PJ, Haseley P, He D, Herrmann M, Hoffman MM, Jeffers TE, Kharchenko PV, Kolasinska-Zwierz P, Kotwaliwale CV, Kumar N, Langley SA, Larschan EN, Latorre I, Libbrecht MW, Lin X, Park R, Pazin MJ, Pham HN, Plachetka A, Qin B, Schwartz YB, Shoresh N, Stempor P, Vielle A, Wang C, Whittle CM, Xue H, Kingston RE, Kim JH, Bernstein BE, Dernburg AF, Pirrotta V, Kuroda MI, Noble WS, Tullius TD, Kellis M, \textbf{MacAlpine DM}*, Strome S, Elgin SC, Liu XS, Lieb JD, Ahringer J, Karpen GH, Park PJ. Comparative analysis of metazoan chromatin organization. Nature. 2014 Aug 28;512(7515):449-52. * Co-corresponding author

\item Nègre N, Brown CD, Ma L, Bristow CA, Miller SW, Wagner U, Kheradpour P, Eaton ML, Loriaux P, Sealfon R, Li Z, Ishii H, Spokony RF, Chen J, Hwang L, Cheng C, Auburn RP, Davis MB, Domanus M, Shah PK, Morrison CA, Zieba J, Suchy S, Senderowicz L, Victorsen A, Bild NA, Grundstad AJ, Hanley D, \textbf{MacAlpine DM}, Mannervik M, Venken K, Bellen H, White R, Gerstein M, Russell S, Grossman RL, Ren B, Posakony JW, Kellis M, White KP. A cis-regulatory map of the Drosophila genome. Nature. 2011 Mar 24;471(7339):527-31. 

\item Kharchenko PV, Alekseyenko AA, Schwartz YB, Minoda A, Riddle NC, Ernst J, Sabo PJ, Larschan E, Gorchakov AA, Gu T, Linder-Basso D, Plachetka A, Shanower G, Tolstorukov MY, Luquette LJ, Xi R, Jung YL, Park RW, Bishop EP, Canfield TK, Sandstrom R, Thurman RE, \textbf{MacAlpine DM}, Stamatoyannopoulos JA, Kellis M, Elgin SC, Kuroda MI, Pirrotta V, Karpen GH, Park PJ. Comprehensive analysis of the chromatin landscape in Drosophila melanogaster. Nature. 2011 Mar 24;471(7339):480-5. 

\item modENCODE Consortium, Roy S, Ernst J, Kharchenko PV, Kheradpour P, Negre N, Eaton ML, Landolin JM, Bristow CA, Ma L, Lin MF, Washietl S, Arshinoff BI, Ay F, Meyer PE, Robine N, Washington NL, Di Stefano L, Berezikov E, Brown CD, Candeias R, Carlson JW, Carr A, Jungreis I, Marbach D, Sealfon R, Tolstorukov MY, Will S, Alekseyenko AA, Artieri C, Booth BW, Brooks AN, Dai Q, Davis CA, Duff MO, Feng X, Gorchakov AA, Gu T, Henikoff JG, Kapranov P, Li R, MacAlpine HK, Malone J, Minoda A, Nordman J, Okamura K, Perry M, Powell SK, Riddle NC, Sakai A, Samsonova A, Sandler JE, Schwartz YB, Sher N, Spokony R, Sturgill D, van Baren M, Wan KH, Yang L, Yu C, Feingold E, Good P, Guyer M, Lowdon R, Ahmad K, Andrews J, Berger B, Brenner SE, Brent MR, Cherbas L, Elgin SC, Gingeras TR, Grossman R, Hoskins RA, Kaufman TC, Kent W, Kuroda MI, Orr-Weaver T, Perrimon N, Pirrotta V, Posakony JW, Ren B, Russell S, Cherbas P, Graveley BR, Lewis S, Micklem G, Oliver B, Park PJ, Celniker SE, Henikoff S, Karpen GH, Lai EC, \textbf{MacAlpine DM}*, Stein LD, White KP, Kellis M. Identification of functional elements and regulatory circuits by Drosophila modENCODE. Science. 2010 Dec 24;330(6012):1787-97. *Co-corresponding author

\end{enumerate}

\item Role of chromatin structure in specifying yeast replication origins

Much progress over the last decade has been made in our understanding of how the local chromatin environment regulates the transcription program.  In contrast, we know comparatively little about how the DNA replication program is regulated by the chromatin landscape.  Our recent work has been focused on understanding how the local chromatin organization impacts the selection and activation of eukaryotic DNA replication origins.  We have found that nucleosome positioning\textsuperscript{a} and ATP-dependent chromatin remodeling\textsuperscript{b,c} at the origin are key determinants regulating the selection and activation of eukaryotic origins.

\begin{enumerate}
\setlength\itemsep{0.35em}

\item Eaton ML, Galani K, Kang S, Bell SP, \textbf{MacAlpine DM}. Conserved nucleosome positioning defines replication origins. Genes Dev. 2010 Apr 15;24(8):748-53. doi: 10.1101/gad.1913210. Epub 2010 Mar 29. 

\item Belsky JA, MacAlpine HK, Lubelsky Y, Hartemink AJ, \textbf{MacAlpine DM}. Genome-wide chromatin footprinting reveals changes in replication origin architecture induced by pre-RC assembly. Genes Dev. 2015 Jan 15;29(2):212-24.

\item Azmi IF, Watanabe S, Maloney MF, Kang S, Belsky JA, \textbf{MacAlpine DM}, Peterson CL, Bell SP. Nucleosomes influence multiple steps during replication initiation. Elife. 2017 Mar 21;6. pii: e22512. doi: 10.7554/eLife.22512. 



\end{enumerate}


\item Establishment and maintenance of the \dros\ DNA replication program

In higher eukaryotes, there is little apparent sequence specificity for ORC and the identification of conserved cis-acting elements that direct origin function has remained elusive.  My research program has used \dros\ to identify the epigenetic determinants of origin selection and function. We have identified activating chromatin marks, nucleosome occupancy, and ATP-dependent chromatin remodelers as being predictive features of ORC binding and origin activation\textsuperscript{a}. We also found that transcription and DNA replication respond to the same chromatin states.  For example, male specific H4K16 hyperacetylation up-regulates transcription and promotes the early replication of the X-chromosome\textsuperscript{b}.  Unlike in mammalian systems, we found that the cell cycle regulated H4K20 monomethylation did not promote helicase loading and origin activation, but rather was essential for maintaining the genomic integrity of late replicating regions of the \dros\ genome\textsuperscript{c}.  Finally, recent work on the loading and distribution of the Mcm2-7 complex revealed that transcription can shape the distribution of the Mcm2-7 complex in late G1, thus providing a possible mechanism(s) for the apparent stochastic activation of replication origins throughout the genomes of higher eukaryotes\textsuperscript{d}.



\begin{enumerate}
\setlength\itemsep{0.35em}

\item Eaton ML, Prinz JA, MacAlpine HK, Tretyakov G, Kharchenko PV, \textbf{MacAlpine DM}. Chromatin signatures of the Drosophila replication program. Genome Res. 2011 Feb;21(2):164-74. 

\item Lubelsky Y, Prinz JA, DeNapoli L, Li Y, Belsky JA, \textbf{MacAlpine DM}. DNA
replication and transcription programs respond to the same chromatin cues. Genome Res. 2014 Jul;24(7):1102-14. doi: 10.1101/gr.160010.113. 

\item Li Y, Armstrong RL, Duronio RJ, \textbf{MacAlpine DM}. Methylation of histone H4 lysine 20 by PR-Set7 ensures the integrity of late replicating sequence domains in Drosophila. Nucleic Acids Res. 2016 Sep 6;44(15):7204-18. doi: 10.1093/nar/gkw333. Epub 2016 Apr 29. 

\item Powell SK, MacAlpine HK, Prinz JA, Li Y, Belsky JA, \textbf{MacAlpine DM}. Dynamic
loading and redistribution of the Mcm2-7 helicase complex through the cell cycle. EMBO J. 2015 Feb 12;34(4):531-43. doi: 10.15252/embj.201488307. Epub 2015 Jan 2. 


\end{enumerate}

\item Collaborative work

\noindent A hallmark of my scientific career has been my willingness to collaborate on a diverse array of projects.  We have been delighted to share our expertise in DNA replication, genomics and computational biology with multiple collaborators.  Recent replication focused collaborations include characterizing the role of DNA methylation and heterochromatin in recruiting ORCA/LRWD1 to mammalian replication origins\textsuperscript{a} and a role for the Mcm2-7 complex in the DNA damage response\textsuperscript{b}.  In collaboration with Zhigou Zhang, we identified non-coding transcription as a disruptor of nucleosome positioning in the absence of the FACT complex\textsuperscript{c}.  Finally, we have been able to apply our computational expertise to identify codon bias as a key effector of oncogenic RAS activity\textsuperscript{d}.  



\begin{enumerate}
\setlength\itemsep{0.35em}


\item Wang Y, Khan A, Marks AB, Smith OK, Giri S, Lin YC, Creager R, \textbf{MacAlpine DM},
Prasanth KV, Aladjem MI, Prasanth SG. Temporal association of ORCA/LRWD1 to
late-firing origins during G1 dictates heterochromatin replication and
organization. Nucleic Acids Res. 2016 Dec 6. 

\item Tsai FL, Vijayraghavan S, Prinz J, MacAlpine HK, \textbf{MacAlpine DM}, Schwacha A.
Mcm2-7 Is an Active Player in the DNA Replication Checkpoint Signaling Cascade
via Proposed Modulation of Its DNA Gate. Mol Cell Biol. 2015 Jun;35(12):2131-43. 
doi: 10.1128/MCB.01357-14. Epub 2015 Apr 13. 

\item Feng J, Gan H, Eaton ML, Zhou H, Li S, Belsky JA, \textbf{MacAlpine DM}, Zhang Z, Li Q.
Noncoding Transcription Is a Driving Force for Nucleosome Instability in spt16
Mutant Cells. Mol Cell Biol. 2016 Jun 15;36(13):1856-67. doi:
10.1128/MCB.00152-16. Print 2016 Jul 1. 


\item Lampson BL, Pershing NL, Prinz JA, Lacsina JR, Marzluff WF, Nicchitta CV,
\textbf{MacAlpine DM}, Counter CM. Rare codons regulate KRas oncogenesis. Curr Biol. 2013 
Jan 7;23(1):70-5. doi: 10.1016/j.cub.2012.11.031. Epub 2012 Dec 13.

\end{enumerate}



\end{enumerate}

\subsection*{Complete List of Published Work in MyBibliography:} 
\medskip

\url{https://www.ncbi.nlm.nih.gov/myncbi/browse/collection/44041939/?sort=date&direction=descending}


%------------------------------------------------------------------------------

\section{Research Support}

\subsection*{Ongoing Research Support}
\medskip

\grantinfo{R01 GM104097}
{MacAlpine (PI)}
{08/09/13--08/08/17}
{\it Chromatin architecture defines DNA replication origins}
{The goal of this project is to understand how the local chromatin architecture impacts the selection and activation of eukaryotic DNA replication origins in \scer.}
{Role: PI}

\bigskip

\grantinfo{R01 GM118551}
{Hartemink (PI)}
{04/01/16--03/31/20}
{\it Exploring the role of dynamic chromatin occupancy in transcriptional regulation}
{The goal of this project is to generate predictive models of cell cycle transcription from chromatin occupancy studies.}
{Role: Co-I}

\bigskip


%------------------------------------------------------------------------------

\subsection*{Completed Research Support}
\medskip

\grantinfo{ACS RSG-11-048-01-DMC}
{MacAlpine (PI)}
{01/01/11--12/31/14}
{\it Defining the Human DNA Replication Program}
{The goal of this project was to establish a comprehensive and genome-wide survey of the human DNA replication program.  We also investigated how the replication program responds to oncogenic transformation.}
{Role: PI}

\bigskip

\grantinfo{NIH U01 HF004279}
{MacAlpine (PI)}
{05/04/07--03/31/12}
{\it The Systematic Identification and Analysis of Replication Origins in Drosophila}
{The major goal of the modENCODE project was to identify all functional DNA elements in a model organism genome.  We  specifically identified and analyzed the sequence elements that direct DNA replication in \dmel.}
{Role: PI}




\end{document}

Chromatin architecture defines DNA replication origins
NIH/NIGMS 1R01GM104097-01
08/09/13-08/08/17
Role: PI
The goal of this project is to understand how the local chromatin architecture impacts the selection and activation of eukaryotic DNA replication origins in S. cerevisiae.  

The role of epigenetics in the formation and consolidation of long-term memories
National University of Singapore
03/01/15-02/29/16
The goal of this project is to profile epigenetic modifications during the the formation and long-term consolidation of memories. 
Role: Co-I

Exploring the Role of Dynamic Chromatin Occupancy in Transcriptional Regulation
NIH/NIGMS R01GM118551 (Hartemink) 04/01/16-03/31/20 1.8 calendar
Role: Co-I
The goal of this project is to generate predictive models of cell cycle transcription from chromatin occupancy studies.


Completed

Defining the Human DNA Replication Program
American Cancer Society RSG-11-048-01-DMC (MacAlpine)
01/01/11-12/31/14 (In a no cost extension for 2015)
Role: PI
The goal of this project is to establish a comprehensive and genome-wide survey of the human DNA replication program.  We will also investigate how the replication program responds to oncogenic transformation. 

The Systematic Identification and Analysis of Replication Origins in Drosophila
NIH/NHGRI U01 HG004279
05/04/07-03/31/12
The major goal of the modENCODE project is to identify all functional DNA elements in a model organism genome.  We are specifically identifying and analyzing the sequences elements that direct DNA replication in Drosophila melanogaster.
